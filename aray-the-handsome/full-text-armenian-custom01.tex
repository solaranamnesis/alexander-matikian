%XeLaTeX
\documentclass{article}
\usepackage{lscape}
\usepackage{fontspec}
\setmainfont{DejaVu Serif}
\newfontfamily{\arm}[Script=Armenian]{DejaVuSans}
\defaultfontfeatures{Scale=MatchLowercase}
\usepackage[dvipsnames]{xcolor}
\usepackage{eso-pic,graphicx}
\usepackage[top=33mm, bottom=37mm, outer=27mm, inner=27mm, landscape]{geometry}
\color{White}
\setlength{\emergencystretch}{45pt}
\AddToShipoutPictureBG{\includegraphics[width=\paperwidth,height=\paperheight]{ara1.jpeg}}
\begin{document}
\begin{titlepage} % Suppresses headers and footers on the title page
	\centering % Centre everything on the title page
	%\scshape % Use small caps for all text on the title page

	%------------------------------------------------
	%	Title
	%------------------------------------------------

	\rule{\textwidth}{1.6pt}\vspace*{-\baselineskip}\vspace*{2pt} % Thick horizontal rule
	\rule{\textwidth}{0.4pt} % Thin horizontal rule
	
	\vspace{1\baselineskip} % Whitespace above the title
	
	{\scshape\Huge \arm{Արայ Գեղեցիկ}}
	
	\vspace{1\baselineskip} % Whitespace above the title

	\rule{\textwidth}{0.4pt}\vspace*{-\baselineskip}\vspace{3.2pt} % Thin horizontal rule
	\rule{\textwidth}{1.6pt} % Thick horizontal rule
	
	\vspace{1\baselineskip} % Whitespace after the title block
	
	%------------------------------------------------
	%	Subtitle
	%------------------------------------------------
	
        {\large \arm{Գրեց Հ. Աղեքսանդր Վ. Մատիկեան}}
 
        \vspace{1.0\baselineskip}
‌
	%------------------------------------------------
	%	Editor(s)
	%------------------------------------------------
        \vspace*{\fill}    

        \vspace{1.0\baselineskip}

        { \arm{Մխիթարեան տպարան}}
        
	\vspace{1\baselineskip}

        {\small\arm{Վիեննա} 1930}
		
	\vspace{0.25\baselineskip} % Whitespace after the title block

        {\scshape\small Solar Anamnesis Edition}% Publication year}
    
	{\scshape\footnotesize Attribution-ShareAlike 4.0 International } % Publisher
\end{titlepage}
\clearpage
\tableofcontents
\clearpage
\large
\section*{\arm{Յառաջաբան}}
\paragraph{}
\arm{Ներկայ Ուսումնասիրութիւնս, որ 1923էն ի վեր „ՀԱՆԴԻՍԻ“ մէջ պարբերաբար հրատարակուեցաւ, ցաւալի է, որ ինծմէ անկախ պատճառներով հազիւ այսօր կրնամ հրապարակ հանել: Թէեւ յապաղմամբս գործը իր միութենէն ոչինչ տուժած է, սակայն դժբախտաբար նոյնը չէ կարելի հաստատել նաեւ տպագրական թերթերուն թուղթին միօրինակութեան նկատմամբ, որոնց իւրաքանչիւրը կարծես պաշտօնը ստանձնած ըլլար Աւստրիոյ տնտեսական տագնապին դրոշմը կրելու իրենց ճակտին: Տպագրական ուղղեւիքներու աոանձին ցանկ մը կաղմել հարկ չեմ տեսներ. միայն մտադիր կ՚ընեմ որ տպագրական 12րդ թերթը մամլոյ տակ դնելու միջոցին սխալմամբ տողերու ետեւառաջութիւն տեղի ունեցած ըլլալուն՝ էջ 179֊180 իմաստի բաւական մեծ խանգարում մտած է. „ՀԱՆԴԻՍԻ“ մէջ արդէն անգամ մը ուղիղ տպուած նախնականը ստանալու համար պէտք է էջ 179֊ի կերջին տողը յաջորդ էջին 5րդ տողէն կերջը զետեղել. յետոյ էջ 6ի „Պղատոմէնի քով“ը կարդալու է „Պղատոնի քով:“

Հոս պատշաճ կը համարիմ յիշատակել երկու գործեր, որոնք ուսումնասիրութեանս ընթացքին անմատչելի մնացին ինծի. մին է Արարատ Ամսագիր, Էջմիածին 1918, ուր Լ. Մելիքսէթ Բեկեան Արալէզներու հայող „Մի՛ քանի խօսք Առլեզի պաշտամունքի մասին“ անունով յօդուած մը հրատարակած է. իսկ միւսն է Dr. Bleichsteiner, Kaukasische Forschungen, 1. Teil, Georgische und Mingrelische Texte, Wien 1919: Առաջնոյն մէջ Մելիքսէթ Բեկեան կը խօսի Աբխազներու „Ալըշկինդր“ շնաստուածներու մասին եւ զրոյց մըն ալ կը հրատարակէ, ուր իբր հերոս կը ներկայանայ Ասլան, որ „բոլոր դեւերը կը սպաննէր. „Օր մը գողունի կերպով մօտեցան եւ կապեցին մետաքսէ ժապաւէններով եւ կուրացընելով՝ սպաննեցին նրան եւ ձգեցին խորը ջրհոր փոսի մէջ: Երբ նրա շները վերադարձան, փնտռեցին իրենց տիրոջը, բայց չկարողացան հանել ջրհորից: Շները երեք օր երեք գիշեր լիզում էին Ասլանին, որի վերադարձրին ե՛ւ կեանքը ե՛ւ տեսողութիւնը.“ անդ 125-126: Այս հետաքրքրական զրոյցին նման նիւթով կը զբաղի նաեւ Dr. Bleichsteinerի թարգմանութեան հրատարակած „Երկու եղբայրներ“ անունով վրական զրոյցը, որ սակայն աւելի սերտ աղերսի մէջ կ՚երեւայ ուսումնասիրութեանս էջ 32-40 ներկայացուած Քոյր եւ եղբայր զրոյցին հետ: Dr. Bleichsteinerի գործը երբ ձեռքս անցաւ, իմ հրատարակածս արդէն տպուած էր: Ջրոյցը երկու մասի կը բաժնուի, որուն երկրորդ մասը յաւելուած է իմ կարծիքովս: Առաջին մասն է բուն հետաքրքրականը, որ ընդհանրապէս հետեւեալն է. Այրի մը որդի մ՚ունէր, որ որսի ժամանակ տեսնելով որ գոմէշ մը եւ անոր ձագը վագրէ մը գիշատուելու վրայ են, կը յարձավի եւ վագրը կը սպաննէ: Գոմէշը ասոր վրայ շնորհակալ կ՚ըլլայ տղուն եւ իբր վարձք իր ձագը կու տայ անոր Քիչ ատենէն տղան կը տեսնէ, որ փոքրիկ գոմէշը հրաշալիք մըն է ահագին ուժով եւ ամէն բանի կարող: Անոր վրայ նստած օդոյ վրայէն թռչելով կու գայ կը հասնի մինչեւ ծով մը, զոր վ՚անցնի ու դեւերու թագաւորութեան տէր վ՚ըլլայ: Բայց իր սխալովը գոմէշը ծովէն ելած վիշապաձգէ մը կը սպաննուի, որ վերջին շունչը չտուած՝ հերոսին կը հրամայէ, որ իր միօը ուտէ, որպէս զի անով իր ոյժը ստանայ եւ ոսկրներն թաղէ, որոնցմէ երկու շնիներ յառաջ կու գան:

Յետոյ կ՚երթայ մայրը կը բերէ, որ վիշապաձկէն խաբուելով՝ տղան սպաննել կ՚ուզէ: Իմ հրատարակած զրոյցիս պէս հոս ալ հերոսին կեանքին դարանակալը (հոօ մայրը) երեք անգամ հիւանդ կը ձեւանայ եւ առողջանալու համար տղան դժուարին ձեռնարկութիւններու կը մղէ, որոնք ի հարկէ իմ հրատարակածէս քիչ մը տարբեր են: Առաջին անգամուն մայր խոզ մը կ՚ուզէ, երկրորդին՝ սիրամարգ մը եւ եռորդիին՝ անմահութեան աղբիւրէն ջուր: Միայն վերջնոյս առթիւ հերոսը Մցիս-Ունախավի (=արեւէն չտեսնուած) աղջկան կը հանդիպի, որ զինքն կը զգուշացընէ, բայց նա միտ չի դներ. ասոր վրայ աղջիկը կախարդական գաւազան մը կու տայ իրեն, որով լեռը երկուքի կը ճեղքէ եւ անմահութեան ջուրէ կ՚առնու, բայց անկից դառնալու միջոցին՝ շնիկները փակուած կը մնան: Մայրը ճարահատ տղուն ոյժը փորձել կ՚ուզէ՝ շղթայով մը ձեռքերը կապելով, երբ չոռորդ անգամուն տղղան չորեքպատիկ շղթան ա՛լ չի՛ կրնար խզել, հոն առանձին տեղ մը պահուըտած վիշապաձուկը դուրս ցատկելով զինքն սպաննել կ՚ուզէ, բայց նոյն միջոցին շնիկներն կու գան եւ իրենց տէրը շղթաներէն կ՚ազատեն եւ վիշապը պատառ պատառ կ՚ընեն: Երկու զրոյցներուն համեմատական նմանութիւնները այնչափ զգալի են եւ յայտնի, որ այս մասին մանրամասնութիւններու իջնալ աւելորդ կը նկատեմ եւ յառաջաբանի մը սահմանէն ալ դուրս:

Գործին ամբողջութեան համար եղած այս երկու փոքրիկ յաւելումներուն կը կցեմ նաեւ ուրիշ կէտ մը, որ ըստ ինքեան իմ մտադրութենէս վրիպած փոքրիկ սխալանք մըն է. այսինքն՝ էջ 233ին „Ատտիսի առասպելէն յայտնի կ՚երեւայ, որ նաեւ Կիւբեղէ Ատտիսի ծննդականը չի գտներ“ խօսքը իր այս ընդհանուր եւ բացասական իմաստով ճիշտ չէ, վասն զի՝ ինչպէս յետոյ մտադիր եղայ ուրիշ տարբերակի մը համաձայն (Arnobius, 5. 14) դիցուհին անդամը ոչ միայն կը գտնէ, այլ եւ առանձին արարողութեամբ ալ կը թաղէ: Թէեւ գիտնականներէն շատերը ինչ ինչ տեսակէտներով Առնոբիոսի դիցաբանական տեղեկութիւններուն ընդհանրապէս կասկածով կը մերձենան, բայց՝ որչափ կ՚երեւայ, իր ըսածները գէթ ինչ ինչ քաղաքներու համար որոշ հիմ մ՚ունին:

„Նիւթերու եւ անուններու ցանկ ին մէջ առնուած են ինչ որ դիցաբանօրէն նշանակութիւն մ՚ունի ուրիշ կարգի նիւթերէ միայն կարեւորները:“

\bigskip

Հեղինակը
}
\clearpage
\section*{\arm{Ներածոիթիին}}
\paragraph{}
\arm{
„Կրօնի ծագումը եւ դիցաբանութիւն“ ուսումնասիրութեան մէջ տուած խոստումս է,
}
\clearpage
\section{\arm{Աջգային Աղբիիրներ Արայի Մասին}}
\subsection{\arm{Գրական աւանդութիւն}}
\paragraph{}
\clearpage
\subsection{\arm{Ժողովրդական աւանդութիւն}}
\paragraph{}
\clearpage
\subsection{\arm{Շամիրամ}}
\paragraph{}
\clearpage
\subsection{\arm{Արալէզները}}
\paragraph{}
\clearpage
\section{\arm{Օտար Աղբիիրներ Արայի Մասին Եի Համեմատոիթիիններ}}
\subsection{\arm{Արայ Գեղեցիկի համապատասխան առասպելներ}}
\paragraph{}
\clearpage
\subsection{\arm{Արայ Գեղեցիկ յոյն մատենագրութեան մէջ}}
\paragraph{}
\clearpage
\subsection{\arm{Արայ անունը եւ իր ծագումը}}
\paragraph{}
\clearpage

\end{document}
